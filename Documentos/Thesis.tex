\documentclass[a4paper,openright,12pt, oneside]{book}
\usepackage[spanish]{babel} 
%\usepackage[latin1]{inputenc}
\usepackage{url}
\usepackage{graphics,graphicx}
\usepackage{algpseudocode}
\usepackage{colortbl}
\usepackage{anysize} %Para los margeness
\usepackage{array}
\usepackage{multirow} % para unir filas
\usepackage{multicol}
\usepackage{dsfont}
\usepackage{blkarray}
\usepackage{amsmath} %para hacer la combinatoria
\usepackage{listings}
\usepackage{enumitem}
\renewcommand{\lstlistingname}{C\'odigo}% Listing -> C\'odigo
\renewcommand{\lstlistlistingname}{List of \lstlistingname s}% List of Listings 
\usepackage{mathtools}
\usepackage{tabularx,ragged2e}
\usepackage{titlesec}
\usepackage{algorithmicx}
\usepackage[capposition=bottom]{floatrow}
\graphicspath{ {/home/emanuel/Tesis/Tesis/Documentos/img/} }
\usepackage[linesnumbered, spanish]{algorithm2e}
\lstset{ %
language=Python,                % choose the language of the code
basicstyle=\footnotesize,       % the size of the fonts that are used for the code
numbers=left,                   % where to put the line-numbers
numberstyle=\footnotesize,      % the size of the fonts that are used for the line-numbers
stepnumber=1,                   % the step between two line-numbers. If it is 1 each line will be numbered
numbersep=5pt,                  % how far the line-numbers are from the code
backgroundcolor=\color{white},  % choose the background color. You must add \usepackage{color}
showspaces=false,               % show spaces adding particular underscores
showstringspaces=false,         % underline spaces within strings
showtabs=false,                 % show tabs within strings adding particular underscores
frame=single,                   % adds a frame around the code
tabsize=2,                      % sets default tabsize to 2 spaces
captionpos=b,                   % sets the caption-position to bottom
breaklines=true,                % sets automatic line breaking
breakatwhitespace=false,        % sets if automatic breaks should only happen at whitespace
mathescape=true,                % para agregar simbolos YOOOOO
escapeinside={\%*}{*)}          % if you want to add a comment within your code
}

\DeclarePairedDelimiter\norm{\lVert}{\rVert}
\DeclareRobustCommand{\orderof}{\ensuremath{\mathcal{O}}}
\DeclareRobustCommand{\matrix}{\ensuremath{\mathcal{M}}}
\newcommand{\prointerno}[2]{\langle #1 \mid #2 \rangle}
\newcommand{\prointernot}[3]{\langle #1 \mid #2 \mid #3 \rangle}
\newcommand{\sii}{\Leftrightarrow}
\newcommand{\reales}{\mathds{R}}
\newcommand{\enteros}{\mathds{Z}}
\newcommand{\nat}{\mathds{N}}
\newcommand{\infinito}{\infty}
\def\blacksquare{\hbox{\vrule width 5pt height 5pt depth 0pt}}
\def\fin{\ \ \ \hbox{}\nolinebreak\hfill $\blacksquare \  \  \  \  $ \par{}\medskip}

\newcommand{\implica}{\Rightarrow}


\makeatletter
\newif\if@borderstar
\def\bordermatrix{\@ifnextchar*{%
\@borderstartrue\@bordermatrix@i}{\@borderstarfalse\@bordermatrix@i*}%
}
\def\@bordermatrix@i*{\@ifnextchar[{\@bordermatrix@ii}{\@bordermatrix@ii[()]}}
\def\@bordermatrix@ii[#1]#2{%
\begingroup
\m@th\@tempdima8.75\p@\setbox\z@\vbox{%
\def\cr{\crcr\noalign{\kern 2\p@\global\let\cr\endline }}%
\ialign {$##$\hfil\kern 2\p@\kern\@tempdima & \thinspace %
\hfil $##$\hfil && \quad\hfil $##$\hfil\crcr\omit\strut %
\hfil\crcr\noalign{\kern -\baselineskip}#2\crcr\omit %
\strut\cr}}%
\setbox\tw@\vbox{\unvcopy\z@\global\setbox\@ne\lastbox}%
\setbox\tw@\hbox{\unhbox\@ne\unskip\global\setbox\@ne\lastbox}%
\setbox\tw@\hbox{%
$\kern\wd\@ne\kern -\@tempdima\left\@firstoftwo#1%
\if@borderstar\kern2pt\else\kern -\wd\@ne\fi%
\global\setbox\@ne\vbox{\box\@ne\if@borderstar\else\kern 2\p@\fi}%
\vcenter{\if@borderstar\else\kern -\ht\@ne\fi%
\unvbox\z@\kern-\if@borderstar2\fi\baselineskip}%
\if@borderstar\kern-2\@tempdima\kern2\p@\else\,\fi\right\@secondoftwo#1 $%
}\null \;\vbox{\kern\ht\@ne\box\tw@}%
\endgroup}
\SetKwRepeat{Do}{do}{while}
\providecommand{\abs}[1]{\lvert#1\rvert}

\bibliographystyle{plain}

\setcounter{tocdepth}{5}
\setcounter{secnumdepth}{5}

\begin{document}

\begin{titlepage}

\begin{center}
\vspace*{-1in}


FACULTAD DE MATEM\'ATICA, ASTRONOM\'IA, F\'ISICA Y COMPUTACI\'ION\\
\vspace*{0.15in}
DEPARTAMENTO DE COMPUTACI\'ON \\
\vspace*{0.6in}
\begin{large}
\end{large}
\vspace*{0.2in}
\begin{Large}
\textbf{Optimizaci\'on del c\'omputo para la resoluci\'on del problema de una y dos part\'iculas en un pozo de potencial usando B-splines} \\
\end{Large}
\vspace*{0.3in}
\begin{large}
Tesis realizada por Emanuel Emilio Lupi para la Licenciatura en Ciencias de la Computaci\'on en la Universidad Nacional de C\'ordoba\end{large}

\vspace*{0.3in}
\rule{80mm}{0.1mm}\\
\vspace*{0.1in}
\begin{large}
Dirigida por: \\
Doctor Nicol\'as Wolovick\\
Licenciado Mariano Garagiola\\
\vspace*{0.1in}
\end{large}
\end{center}
\end{titlepage}

\mbox{}
\thispagestyle{empty}
\pagenumbering{arabic}

\chapter*{Agradecimientos} % si no queremos que a\~nada la palabra "Capitulo"
\addcontentsline{toc}{chapter}{Agradecimientos} % si queremos que aparezca en el \'indice
\markboth{AGRADECIMIENTOS}{AGRADECIMIENTOS} % encabezado

\begin{itemize}
\item A mi familia, quienes siempre ser\'an los primeros agradecidos en
cualquier logro de mi vida por formarme como persona, por el amor y el apoyo incondicional que siempre me brindan.

\item A mis directores Nicol\'as y Mariano por la calidez humana y la gran ayuda que siempre me prestaron a lo largo del proyecto.

\item A mis amigos y compa\~neros Juan, Kevin, Fernando, Franco, Emiliano, Maxi, Eric, H\'ector, El\'ias, Pablo, Ezequiel, Agust\'in, Leandro, Emanuel, Joaqu\'in, Federico, Gonzalo, Alan, Joaco, Giovanni, Marco, Demetrio, Marcelo y tantos mas.

\end{itemize}

\chapter*{\hspace{0.65cm}Resumen} % si no queremos que a\~nada la palabra "Capitulo"
\addcontentsline{toc}{chapter}{Resumen} % si queremos que aparezca en el \'indice
\markboth{RESUMEN}{RESUMEN} % encabezado

La ecuaci\'on de Schr\"odinger independiente del tiempo predice que las funciones de onda pueden tener la forma de ondas estacionarias, denominados estados estacionarios (tambi\'en llamados "orbitales", como en los orbitales at\'omicos o los orbitales moleculares). A partir de los estados estacionarios es posible encontrar los estados dependientes del tiempo de un sistema f\'isico. La ecuaci\'on de Schr\"odinger independiente del tiempo es la ecuaci\'on que describe los estados estacionarios.

En una gran variedad de situaciones f\'isicas, por ejemplo cuando se quiere conocer la funci\'on de onda espacial, la ecuaci\'on de Schr\"odinger es una ecuaci\'on diferencial en derivadas parciales de segundo orden

\begin{displaymath}
    E\psi(r) = \left[\frac{-\hbar^2}{2\mu} \nabla^2 + V(r)\right]\psi(r)
\end{displaymath}

la cual, a su vez es una ecuaci\'on de autofunciones y autovalores, donde las autofunciones son los estados estacionarios de la ecuaci\'on y los autovalores son las energ\'ias asociadas a cada autofunci\'on.

Las ecuaciones diferenciales de segundo orden son complejas de resolver ya que no existe una forma anal\'itica para la soluci\'on de una ecuaci\'on arbitraria, es por eso que en la mayor\'ia de los casos se buscan soluciones aproximadas. Uno de los m\'etodos usados para resolver la ecuaci\'on de Schr\"odinger independiente del tiempo es el m\'etodo variacional de Rayleigh-Ritz. Este m\'etodo consiste en escribir una representaci\'on matricial de la ecuaci\'on diferencial, a partir de un conjunto de funciones, y luego calcular los autovalores y autovectores de dicha matriz.

En este trabajo se recibi\'o una implementaci\'on del m\'etodo variacional de Rayleigh-Ritz, en el que se usaba el conjunto de funciones conocido como B-splines, el cual se optimiz\'o utilizando estructuras de datos y modificaciones algor\'itmicas m\'as eficientes.


\tableofcontents % indice de contenidos

\setcounter{chapter}{1}
\input{chapters/Chapter1}

\stepcounter{chapter}
\setcounter{section}{0}
\input{chapters/Chapter2}

\stepcounter{chapter}
\setcounter{section}{0}
\input{chapters/Chapter3}

\stepcounter{chapter}
\setcounter{section}{0}
\input{chapters/Chapter4}

\stepcounter{chapter}
\setcounter{section}{0}
\input{chapters/Chapter5}

\stepcounter{chapter}
\setcounter{section}{0}
\input{chapters/Chapter6}


\begin{thebibliography}{X}

\bibitem{SPARSKIT}
  SPARSKIT 
  \emph{working note 50: Distributed sparse data structures for linear algebra operations},
  Tech. Rep. CS 92-169, Computer Science Department, University of Tennessee, Knoxville, TN.
  1992.

\bibitem{LAPACK}
  LAPACK 
  \emph{A basic tool kit for sparse matrix computation, Tech. Rep. CSRD TR 1029, CSRD},
  University of Illinois, Urbana, IL
  1990

\bibitem{IRAM}
    R. B. Lehoucq \& D. C. Sorensen (1996). 
    \emph{"Deflation Techniques for an Implicitly Restarted Arnoldi Iteration"}

\bibitem{JDS}
  Krylov,
  \emph{SFLubspace methods on supercomputers},
  SIAM J. Sci. Statist. Comput.
  10 (1989), pp. 1200-1232.

\bibitem{SKYMAT}
  I. S. DUFF, A. M. ERISMAN, AND J.K.REID,
  \emph{Direct methods for sparse matrices},
  Oxford University Press, London
  1986

\bibitem{LINPACK}
  J. DONGARRA, C. MOLER, J. BUNCH, AND G. STEWART
  \emph{LINPACK Users' Guide, SIAM},
  Philadelphia
  1979.

\bibitem{MELHEM}
  R. MELHEM
  \emph{Toward efficient implementation of preconditioned conjugate gradient methods on vector supercomputers},
  Internat. J. Supercomput. Appls., 1 (1987), pp. 77-98

\bibitem{Cullum}
  Cullum; Willoughby. 
  \emph{Lanczos Algorithms for Large Symmetric Eigenvalue Computations. 1.} 
  ISBN 0-8176-3058-9.

\bibitem{booksaad}
  Yousef Saad. 
  \emph{Numerical Methods for Large Eigenvalue Problems.} 
  ISBN 0-470-21820-7.
  %387

\bibitem{FORMHESS}
    W. E. Arnoldi. 
    \emph{The principle of minimized iterations in the solution of the matrix eigenvalue problem.} 
    Quart. Appl. Math., 9:17-29, 1951.
\bibitem{SaadNumMeth}
    Y. Saad. 
    \emph{Numerical Methods for Large Eigenvalue Problems.}
    Halsted Press, New York, 1992

\bibitem{Parlett}
    B. N. Parlett. 
    \emph{The Symmetric Eigenvalue Problem.}
    Prentice-Hall, Englewood Cliffs, NJ, 1980. 
    Reprinted as Classics in Applied Mathematics 20, SIAM, Philadelphia, 1997.
    %353

\bibitem{Householder}
    G. Golub and C. Van Loan. 
    \emph{Matrix Computations.}
    The Johns Hopkins University Press, Baltimore, third edition, 1996.

\bibitem{Householder2}
    H. F. Walker. 
    \emph{Implementation of the GMRES method using Householder transformations.}
    SIAM J. Sci. Statist. Comput., 9:152-163, 1988.

\bibitem{libromagico}
    \emph{Templates for the Solution of Algebraic Eigenvalue Problems}
    Zhaojun Bai, James Demmel, Jack Dongarra, Axel Ruhe, and Henk van der Vorst

\bibitem{ARPACK}
    \emph{http://www.caam.rice.edu/software/ARPACK/}

\bibitem{macdonald}
    Phys. Rev. 43, 830-833 (1933)

\bibitem{IRLM}
    D. Calvetti, L. Reichel, and D.C. Sorensen (1994). 
    \emph{An Implicitly Restarted Lanczos Method for Large Symmetric Eigenvalue Problems}. 
    Electronic Transactions on Numerical Analysis 2: 1-21.

\bibitem{QR}
    J.G.F. Francis.
    \emph {The QR Transformation, I}.
    The Computer Journal, vol. 4, no. 3, pages 265-271 (1961, received Oct 1959) 

\bibitem{PAIGE}
    Paige, C.C. 
    \emph{Computational variants of the Lanczos method for the eigenproblem}.
    J. Inst. Maths Applics 10, 373-381 (1972).

\bibitem{Demmel}
    J. Demmel. 
    \emph{Applied Numerical Linear Algebra}.
    SIAM, Philadelphia, 1997.

\bibitem{FLYNN_TAXONOMY}
  Flynn
  \emph{Some Computer Organizations and Their Effectiveness}

\bibitem{latencia_grafico}
  Systems Performance: Enterprise and the Cloud
  \emph{http://www.brendangregg.com/sysperfbook.html}

\end{thebibliography}

%\bibliographystyle{Classes/CUEDbiblio}
%\bibliographystyle{Classes/jmb}
%\bibliographystyle{Classes/jmb} % bibliography style

%\renewcommand{\bibname}{References} % changes default name Bibliography to References
\bibdata{Thesis}
%\bibliography{./References} % References file

\end{document}
